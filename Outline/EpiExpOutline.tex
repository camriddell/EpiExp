\documentclass[12pt,notitlepage,onecolumn,aps,pra]{article}


    
\usepackage[T1]{fontenc}
\usepackage{graphicx}
% We will generate all images so they have a width \maxwidth. This means
% that they will get their normal width if they fit onto the page, but
% are scaled down if they would overflow the margins.
\makeatletter
\def\maxwidth{\ifdim\Gin@nat@width>\linewidth\linewidth
\else\Gin@nat@width\fi}
\makeatother
\let\Oldincludegraphics\includegraphics
% Set max figure width to be 80% of text width, for now hardcoded.
\renewcommand{\includegraphics}[1]{\Oldincludegraphics[width=.8\maxwidth]{#1}}
% Ensure that by default, figures have no caption (until we provide a
% proper Figure object with a Caption API and a way to capture that
% in the conversion process - todo).
\usepackage{caption}
\usepackage{adjustbox} % Used to constrain images to a maximum size
\usepackage{xcolor} % Allow colors to be defined
\usepackage{enumerate} % Needed for markdown enumerations to work
\usepackage{geometry} % Used to adjust the document margins
\usepackage{amsmath} % Equations
\usepackage{amssymb} % Equations
\usepackage{textcomp} % defines textquotesingle
% Hack from http://tex.stackexchange.com/a/47451/13684:
\AtBeginDocument{%
    \def\PYZsq{\textquotesingle}% Upright quotes in Pygmentized code
}
\usepackage{upquote} % Upright quotes for verbatim code
\usepackage{eurosym} % defines \euro
\usepackage[mathletters]{ucs} % Extended unicode (utf-8) support
\usepackage[utf8x]{inputenc} % Allow utf-8 characters in the tex document
\usepackage{fancyvrb} % verbatim replacement that allows latex
\usepackage{grffile} % extends the file name processing of package graphics
                     % to support a larger range
% The hyperref package gives us a pdf with properly built
% internal navigation ('pdf bookmarks' for the table of contents,
% internal cross-reference links, web links for URLs, etc.)
\usepackage{hyperref}
\usepackage{natbib}
\usepackage{booktabs}  % table support for pandoc > 1.12.2
\usepackage[inline]{enumitem} % IRkernel/repr support (it uses the enumerate* environment)
\usepackage[normalem]{ulem} % ulem is needed to support strikethroughs (\sout)
                            % normalem makes italics be italics, not underlines
\usepackage{braket}

\usepackage{rotating}
\usepackage{threeparttable}
\usepackage{subcaption}



    
    % Colors for the hyperref package
    \definecolor{urlcolor}{rgb}{0,.145,.698}
    \definecolor{linkcolor}{rgb}{.71,0.21,0.01}
    \definecolor{citecolor}{rgb}{.12,.54,.11}

    % ANSI colors
    \definecolor{ansi-black}{HTML}{3E424D}
    \definecolor{ansi-black-intense}{HTML}{282C36}
    \definecolor{ansi-red}{HTML}{E75C58}
    \definecolor{ansi-red-intense}{HTML}{B22B31}
    \definecolor{ansi-green}{HTML}{00A250}
    \definecolor{ansi-green-intense}{HTML}{007427}
    \definecolor{ansi-yellow}{HTML}{DDB62B}
    \definecolor{ansi-yellow-intense}{HTML}{B27D12}
    \definecolor{ansi-blue}{HTML}{208FFB}
    \definecolor{ansi-blue-intense}{HTML}{0065CA}
    \definecolor{ansi-magenta}{HTML}{D160C4}
    \definecolor{ansi-magenta-intense}{HTML}{A03196}
    \definecolor{ansi-cyan}{HTML}{60C6C8}
    \definecolor{ansi-cyan-intense}{HTML}{258F8F}
    \definecolor{ansi-white}{HTML}{C5C1B4}
    \definecolor{ansi-white-intense}{HTML}{A1A6B2}
    \definecolor{ansi-default-inverse-fg}{HTML}{FFFFFF}
    \definecolor{ansi-default-inverse-bg}{HTML}{000000}

    % commands and environments needed by pandoc snippets
    % extracted from the output of `pandoc -s`
    \providecommand{\tightlist}{%
      \setlength{\itemsep}{0pt}\setlength{\parskip}{0pt}}
    \DefineVerbatimEnvironment{Highlighting}{Verbatim}{commandchars=\\\{\}}
    % Add ',fontsize=\small' for more characters per line
    \newenvironment{Shaded}{}{}
    \newcommand{\KeywordTok}[1]{\textcolor[rgb]{0.00,0.44,0.13}{\textbf{{#1}}}}
    \newcommand{\DataTypeTok}[1]{\textcolor[rgb]{0.56,0.13,0.00}{{#1}}}
    \newcommand{\DecValTok}[1]{\textcolor[rgb]{0.25,0.63,0.44}{{#1}}}
    \newcommand{\BaseNTok}[1]{\textcolor[rgb]{0.25,0.63,0.44}{{#1}}}
    \newcommand{\FloatTok}[1]{\textcolor[rgb]{0.25,0.63,0.44}{{#1}}}
    \newcommand{\CharTok}[1]{\textcolor[rgb]{0.25,0.44,0.63}{{#1}}}
    \newcommand{\StringTok}[1]{\textcolor[rgb]{0.25,0.44,0.63}{{#1}}}
    \newcommand{\CommentTok}[1]{\textcolor[rgb]{0.38,0.63,0.69}{\textit{{#1}}}}
    \newcommand{\OtherTok}[1]{\textcolor[rgb]{0.00,0.44,0.13}{{#1}}}
    \newcommand{\AlertTok}[1]{\textcolor[rgb]{1.00,0.00,0.00}{\textbf{{#1}}}}
    \newcommand{\FunctionTok}[1]{\textcolor[rgb]{0.02,0.16,0.49}{{#1}}}
    \newcommand{\RegionMarkerTok}[1]{{#1}}
    \newcommand{\ErrorTok}[1]{\textcolor[rgb]{1.00,0.00,0.00}{\textbf{{#1}}}}
    \newcommand{\NormalTok}[1]{{#1}}
    
    % Additional commands for more recent versions of Pandoc
    \newcommand{\ConstantTok}[1]{\textcolor[rgb]{0.53,0.00,0.00}{{#1}}}
    \newcommand{\SpecialCharTok}[1]{\textcolor[rgb]{0.25,0.44,0.63}{{#1}}}
    \newcommand{\VerbatimStringTok}[1]{\textcolor[rgb]{0.25,0.44,0.63}{{#1}}}
    \newcommand{\SpecialStringTok}[1]{\textcolor[rgb]{0.73,0.40,0.53}{{#1}}}
    \newcommand{\ImportTok}[1]{{#1}}
    \newcommand{\DocumentationTok}[1]{\textcolor[rgb]{0.73,0.13,0.13}{\textit{{#1}}}}
    \newcommand{\AnnotationTok}[1]{\textcolor[rgb]{0.38,0.63,0.69}{\textbf{\textit{{#1}}}}}
    \newcommand{\CommentVarTok}[1]{\textcolor[rgb]{0.38,0.63,0.69}{\textbf{\textit{{#1}}}}}
    \newcommand{\VariableTok}[1]{\textcolor[rgb]{0.10,0.09,0.49}{{#1}}}
    \newcommand{\ControlFlowTok}[1]{\textcolor[rgb]{0.00,0.44,0.13}{\textbf{{#1}}}}
    \newcommand{\OperatorTok}[1]{\textcolor[rgb]{0.40,0.40,0.40}{{#1}}}
    \newcommand{\BuiltInTok}[1]{{#1}}
    \newcommand{\ExtensionTok}[1]{{#1}}
    \newcommand{\PreprocessorTok}[1]{\textcolor[rgb]{0.74,0.48,0.00}{{#1}}}
    \newcommand{\AttributeTok}[1]{\textcolor[rgb]{0.49,0.56,0.16}{{#1}}}
    \newcommand{\InformationTok}[1]{\textcolor[rgb]{0.38,0.63,0.69}{\textbf{\textit{{#1}}}}}
    \newcommand{\WarningTok}[1]{\textcolor[rgb]{0.38,0.63,0.69}{\textbf{\textit{{#1}}}}}
    
    
    % Define a nice break command that doesn't care if a line doesn't already
    % exist.
    \def\br{\hspace*{\fill} \\* }
    % Math Jax compatibility definitions
    \def\gt{>}
    \def\lt{<}
    \let\Oldtex\TeX
    \let\Oldlatex\LaTeX
    \renewcommand{\TeX}{\textrm{\Oldtex}}
    \renewcommand{\LaTeX}{\textrm{\Oldlatex}}
    % Document parameters
    % Document title
    
    
    
    
% Pygments definitions
\makeatletter
\def\PY@reset{\let\PY@it=\relax \let\PY@bf=\relax%
    \let\PY@ul=\relax \let\PY@tc=\relax%
    \let\PY@bc=\relax \let\PY@ff=\relax}
\def\PY@tok#1{\csname PY@tok@#1\endcsname}
\def\PY@toks#1+{\ifx\relax#1\empty\else%
    \PY@tok{#1}\expandafter\PY@toks\fi}
\def\PY@do#1{\PY@bc{\PY@tc{\PY@ul{%
    \PY@it{\PY@bf{\PY@ff{#1}}}}}}}
\def\PY#1#2{\PY@reset\PY@toks#1+\relax+\PY@do{#2}}

\expandafter\def\csname PY@tok@w\endcsname{\def\PY@tc##1{\textcolor[rgb]{0.73,0.73,0.73}{##1}}}
\expandafter\def\csname PY@tok@c\endcsname{\let\PY@it=\textit\def\PY@tc##1{\textcolor[rgb]{0.25,0.50,0.50}{##1}}}
\expandafter\def\csname PY@tok@cp\endcsname{\def\PY@tc##1{\textcolor[rgb]{0.74,0.48,0.00}{##1}}}
\expandafter\def\csname PY@tok@k\endcsname{\let\PY@bf=\textbf\def\PY@tc##1{\textcolor[rgb]{0.00,0.50,0.00}{##1}}}
\expandafter\def\csname PY@tok@kp\endcsname{\def\PY@tc##1{\textcolor[rgb]{0.00,0.50,0.00}{##1}}}
\expandafter\def\csname PY@tok@kt\endcsname{\def\PY@tc##1{\textcolor[rgb]{0.69,0.00,0.25}{##1}}}
\expandafter\def\csname PY@tok@o\endcsname{\def\PY@tc##1{\textcolor[rgb]{0.40,0.40,0.40}{##1}}}
\expandafter\def\csname PY@tok@ow\endcsname{\let\PY@bf=\textbf\def\PY@tc##1{\textcolor[rgb]{0.67,0.13,1.00}{##1}}}
\expandafter\def\csname PY@tok@nb\endcsname{\def\PY@tc##1{\textcolor[rgb]{0.00,0.50,0.00}{##1}}}
\expandafter\def\csname PY@tok@nf\endcsname{\def\PY@tc##1{\textcolor[rgb]{0.00,0.00,1.00}{##1}}}
\expandafter\def\csname PY@tok@nc\endcsname{\let\PY@bf=\textbf\def\PY@tc##1{\textcolor[rgb]{0.00,0.00,1.00}{##1}}}
\expandafter\def\csname PY@tok@nn\endcsname{\let\PY@bf=\textbf\def\PY@tc##1{\textcolor[rgb]{0.00,0.00,1.00}{##1}}}
\expandafter\def\csname PY@tok@ne\endcsname{\let\PY@bf=\textbf\def\PY@tc##1{\textcolor[rgb]{0.82,0.25,0.23}{##1}}}
\expandafter\def\csname PY@tok@nv\endcsname{\def\PY@tc##1{\textcolor[rgb]{0.10,0.09,0.49}{##1}}}
\expandafter\def\csname PY@tok@no\endcsname{\def\PY@tc##1{\textcolor[rgb]{0.53,0.00,0.00}{##1}}}
\expandafter\def\csname PY@tok@nl\endcsname{\def\PY@tc##1{\textcolor[rgb]{0.63,0.63,0.00}{##1}}}
\expandafter\def\csname PY@tok@ni\endcsname{\let\PY@bf=\textbf\def\PY@tc##1{\textcolor[rgb]{0.60,0.60,0.60}{##1}}}
\expandafter\def\csname PY@tok@na\endcsname{\def\PY@tc##1{\textcolor[rgb]{0.49,0.56,0.16}{##1}}}
\expandafter\def\csname PY@tok@nt\endcsname{\let\PY@bf=\textbf\def\PY@tc##1{\textcolor[rgb]{0.00,0.50,0.00}{##1}}}
\expandafter\def\csname PY@tok@nd\endcsname{\def\PY@tc##1{\textcolor[rgb]{0.67,0.13,1.00}{##1}}}
\expandafter\def\csname PY@tok@s\endcsname{\def\PY@tc##1{\textcolor[rgb]{0.73,0.13,0.13}{##1}}}
\expandafter\def\csname PY@tok@sd\endcsname{\let\PY@it=\textit\def\PY@tc##1{\textcolor[rgb]{0.73,0.13,0.13}{##1}}}
\expandafter\def\csname PY@tok@si\endcsname{\let\PY@bf=\textbf\def\PY@tc##1{\textcolor[rgb]{0.73,0.40,0.53}{##1}}}
\expandafter\def\csname PY@tok@se\endcsname{\let\PY@bf=\textbf\def\PY@tc##1{\textcolor[rgb]{0.73,0.40,0.13}{##1}}}
\expandafter\def\csname PY@tok@sr\endcsname{\def\PY@tc##1{\textcolor[rgb]{0.73,0.40,0.53}{##1}}}
\expandafter\def\csname PY@tok@ss\endcsname{\def\PY@tc##1{\textcolor[rgb]{0.10,0.09,0.49}{##1}}}
\expandafter\def\csname PY@tok@sx\endcsname{\def\PY@tc##1{\textcolor[rgb]{0.00,0.50,0.00}{##1}}}
\expandafter\def\csname PY@tok@m\endcsname{\def\PY@tc##1{\textcolor[rgb]{0.40,0.40,0.40}{##1}}}
\expandafter\def\csname PY@tok@gh\endcsname{\let\PY@bf=\textbf\def\PY@tc##1{\textcolor[rgb]{0.00,0.00,0.50}{##1}}}
\expandafter\def\csname PY@tok@gu\endcsname{\let\PY@bf=\textbf\def\PY@tc##1{\textcolor[rgb]{0.50,0.00,0.50}{##1}}}
\expandafter\def\csname PY@tok@gd\endcsname{\def\PY@tc##1{\textcolor[rgb]{0.63,0.00,0.00}{##1}}}
\expandafter\def\csname PY@tok@gi\endcsname{\def\PY@tc##1{\textcolor[rgb]{0.00,0.63,0.00}{##1}}}
\expandafter\def\csname PY@tok@gr\endcsname{\def\PY@tc##1{\textcolor[rgb]{1.00,0.00,0.00}{##1}}}
\expandafter\def\csname PY@tok@ge\endcsname{\let\PY@it=\textit}
\expandafter\def\csname PY@tok@gs\endcsname{\let\PY@bf=\textbf}
\expandafter\def\csname PY@tok@gp\endcsname{\let\PY@bf=\textbf\def\PY@tc##1{\textcolor[rgb]{0.00,0.00,0.50}{##1}}}
\expandafter\def\csname PY@tok@go\endcsname{\def\PY@tc##1{\textcolor[rgb]{0.53,0.53,0.53}{##1}}}
\expandafter\def\csname PY@tok@gt\endcsname{\def\PY@tc##1{\textcolor[rgb]{0.00,0.27,0.87}{##1}}}
\expandafter\def\csname PY@tok@err\endcsname{\def\PY@bc##1{\setlength{\fboxsep}{0pt}\fcolorbox[rgb]{1.00,0.00,0.00}{1,1,1}{\strut ##1}}}
\expandafter\def\csname PY@tok@kc\endcsname{\let\PY@bf=\textbf\def\PY@tc##1{\textcolor[rgb]{0.00,0.50,0.00}{##1}}}
\expandafter\def\csname PY@tok@kd\endcsname{\let\PY@bf=\textbf\def\PY@tc##1{\textcolor[rgb]{0.00,0.50,0.00}{##1}}}
\expandafter\def\csname PY@tok@kn\endcsname{\let\PY@bf=\textbf\def\PY@tc##1{\textcolor[rgb]{0.00,0.50,0.00}{##1}}}
\expandafter\def\csname PY@tok@kr\endcsname{\let\PY@bf=\textbf\def\PY@tc##1{\textcolor[rgb]{0.00,0.50,0.00}{##1}}}
\expandafter\def\csname PY@tok@bp\endcsname{\def\PY@tc##1{\textcolor[rgb]{0.00,0.50,0.00}{##1}}}
\expandafter\def\csname PY@tok@fm\endcsname{\def\PY@tc##1{\textcolor[rgb]{0.00,0.00,1.00}{##1}}}
\expandafter\def\csname PY@tok@vc\endcsname{\def\PY@tc##1{\textcolor[rgb]{0.10,0.09,0.49}{##1}}}
\expandafter\def\csname PY@tok@vg\endcsname{\def\PY@tc##1{\textcolor[rgb]{0.10,0.09,0.49}{##1}}}
\expandafter\def\csname PY@tok@vi\endcsname{\def\PY@tc##1{\textcolor[rgb]{0.10,0.09,0.49}{##1}}}
\expandafter\def\csname PY@tok@vm\endcsname{\def\PY@tc##1{\textcolor[rgb]{0.10,0.09,0.49}{##1}}}
\expandafter\def\csname PY@tok@sa\endcsname{\def\PY@tc##1{\textcolor[rgb]{0.73,0.13,0.13}{##1}}}
\expandafter\def\csname PY@tok@sb\endcsname{\def\PY@tc##1{\textcolor[rgb]{0.73,0.13,0.13}{##1}}}
\expandafter\def\csname PY@tok@sc\endcsname{\def\PY@tc##1{\textcolor[rgb]{0.73,0.13,0.13}{##1}}}
\expandafter\def\csname PY@tok@dl\endcsname{\def\PY@tc##1{\textcolor[rgb]{0.73,0.13,0.13}{##1}}}
\expandafter\def\csname PY@tok@s2\endcsname{\def\PY@tc##1{\textcolor[rgb]{0.73,0.13,0.13}{##1}}}
\expandafter\def\csname PY@tok@sh\endcsname{\def\PY@tc##1{\textcolor[rgb]{0.73,0.13,0.13}{##1}}}
\expandafter\def\csname PY@tok@s1\endcsname{\def\PY@tc##1{\textcolor[rgb]{0.73,0.13,0.13}{##1}}}
\expandafter\def\csname PY@tok@mb\endcsname{\def\PY@tc##1{\textcolor[rgb]{0.40,0.40,0.40}{##1}}}
\expandafter\def\csname PY@tok@mf\endcsname{\def\PY@tc##1{\textcolor[rgb]{0.40,0.40,0.40}{##1}}}
\expandafter\def\csname PY@tok@mh\endcsname{\def\PY@tc##1{\textcolor[rgb]{0.40,0.40,0.40}{##1}}}
\expandafter\def\csname PY@tok@mi\endcsname{\def\PY@tc##1{\textcolor[rgb]{0.40,0.40,0.40}{##1}}}
\expandafter\def\csname PY@tok@il\endcsname{\def\PY@tc##1{\textcolor[rgb]{0.40,0.40,0.40}{##1}}}
\expandafter\def\csname PY@tok@mo\endcsname{\def\PY@tc##1{\textcolor[rgb]{0.40,0.40,0.40}{##1}}}
\expandafter\def\csname PY@tok@ch\endcsname{\let\PY@it=\textit\def\PY@tc##1{\textcolor[rgb]{0.25,0.50,0.50}{##1}}}
\expandafter\def\csname PY@tok@cm\endcsname{\let\PY@it=\textit\def\PY@tc##1{\textcolor[rgb]{0.25,0.50,0.50}{##1}}}
\expandafter\def\csname PY@tok@cpf\endcsname{\let\PY@it=\textit\def\PY@tc##1{\textcolor[rgb]{0.25,0.50,0.50}{##1}}}
\expandafter\def\csname PY@tok@c1\endcsname{\let\PY@it=\textit\def\PY@tc##1{\textcolor[rgb]{0.25,0.50,0.50}{##1}}}
\expandafter\def\csname PY@tok@cs\endcsname{\let\PY@it=\textit\def\PY@tc##1{\textcolor[rgb]{0.25,0.50,0.50}{##1}}}

\def\PYZbs{\char`\\}
\def\PYZus{\char`\_}
\def\PYZob{\char`\{}
\def\PYZcb{\char`\}}
\def\PYZca{\char`\^}
\def\PYZam{\char`\&}
\def\PYZlt{\char`\<}
\def\PYZgt{\char`\>}
\def\PYZsh{\char`\#}
\def\PYZpc{\char`\%}
\def\PYZdl{\char`\$}
\def\PYZhy{\char`\-}
\def\PYZsq{\char`\'}
\def\PYZdq{\char`\"}
\def\PYZti{\char`\~}
% for compatibility with earlier versions
\def\PYZat{@}
\def\PYZlb{[}
\def\PYZrb{]}
\makeatother


    % For linebreaks inside Verbatim environment from package fancyvrb. 
    \makeatletter
        \newbox\Wrappedcontinuationbox 
        \newbox\Wrappedvisiblespacebox 
        \newcommand*\Wrappedvisiblespace {\textcolor{red}{\textvisiblespace}} 
        \newcommand*\Wrappedcontinuationsymbol {\textcolor{red}{\llap{\tiny$\m@th\hookrightarrow$}}} 
        \newcommand*\Wrappedcontinuationindent {3ex } 
        \newcommand*\Wrappedafterbreak {\kern\Wrappedcontinuationindent\copy\Wrappedcontinuationbox} 
        % Take advantage of the already applied Pygments mark-up to insert 
        % potential linebreaks for TeX processing. 
        %        {, <, #, %, $, ' and ": go to next line. 
        %        _, }, ^, &, >, - and ~: stay at end of broken line. 
        % Use of \textquotesingle for straight quote. 
        \newcommand*\Wrappedbreaksatspecials {% 
            \def\PYGZus{\discretionary{\char`\_}{\Wrappedafterbreak}{\char`\_}}% 
            \def\PYGZob{\discretionary{}{\Wrappedafterbreak\char`\{}{\char`\{}}% 
            \def\PYGZcb{\discretionary{\char`\}}{\Wrappedafterbreak}{\char`\}}}% 
            \def\PYGZca{\discretionary{\char`\^}{\Wrappedafterbreak}{\char`\^}}% 
            \def\PYGZam{\discretionary{\char`\&}{\Wrappedafterbreak}{\char`\&}}% 
            \def\PYGZlt{\discretionary{}{\Wrappedafterbreak\char`\<}{\char`\<}}% 
            \def\PYGZgt{\discretionary{\char`\>}{\Wrappedafterbreak}{\char`\>}}% 
            \def\PYGZsh{\discretionary{}{\Wrappedafterbreak\char`\#}{\char`\#}}% 
            \def\PYGZpc{\discretionary{}{\Wrappedafterbreak\char`\%}{\char`\%}}% 
            \def\PYGZdl{\discretionary{}{\Wrappedafterbreak\char`\$}{\char`\$}}% 
            \def\PYGZhy{\discretionary{\char`\-}{\Wrappedafterbreak}{\char`\-}}% 
            \def\PYGZsq{\discretionary{}{\Wrappedafterbreak\textquotesingle}{\textquotesingle}}% 
            \def\PYGZdq{\discretionary{}{\Wrappedafterbreak\char`\"}{\char`\"}}% 
            \def\PYGZti{\discretionary{\char`\~}{\Wrappedafterbreak}{\char`\~}}% 
        } 
        % Some characters . , ; ? ! / are not pygmentized. 
        % This macro makes them "active" and they will insert potential linebreaks 
        \newcommand*\Wrappedbreaksatpunct {% 
            \lccode`\~`\.\lowercase{\def~}{\discretionary{\hbox{\char`\.}}{\Wrappedafterbreak}{\hbox{\char`\.}}}% 
            \lccode`\~`\,\lowercase{\def~}{\discretionary{\hbox{\char`\,}}{\Wrappedafterbreak}{\hbox{\char`\,}}}% 
            \lccode`\~`\;\lowercase{\def~}{\discretionary{\hbox{\char`\;}}{\Wrappedafterbreak}{\hbox{\char`\;}}}% 
            \lccode`\~`\:\lowercase{\def~}{\discretionary{\hbox{\char`\:}}{\Wrappedafterbreak}{\hbox{\char`\:}}}% 
            \lccode`\~`\?\lowercase{\def~}{\discretionary{\hbox{\char`\?}}{\Wrappedafterbreak}{\hbox{\char`\?}}}% 
            \lccode`\~`\!\lowercase{\def~}{\discretionary{\hbox{\char`\!}}{\Wrappedafterbreak}{\hbox{\char`\!}}}% 
            \lccode`\~`\/\lowercase{\def~}{\discretionary{\hbox{\char`\/}}{\Wrappedafterbreak}{\hbox{\char`\/}}}% 
            \catcode`\.\active
            \catcode`\,\active 
            \catcode`\;\active
            \catcode`\:\active
            \catcode`\?\active
            \catcode`\!\active
            \catcode`\/\active 
            \lccode`\~`\~ 	
        }
    \makeatother

    \let\OriginalVerbatim=\Verbatim
    \makeatletter
    \renewcommand{\Verbatim}[1][1]{%
        %\parskip\z@skip
        \sbox\Wrappedcontinuationbox {\Wrappedcontinuationsymbol}%
        \sbox\Wrappedvisiblespacebox {\FV@SetupFont\Wrappedvisiblespace}%
        \def\FancyVerbFormatLine ##1{\hsize\linewidth
            \vtop{\raggedright\hyphenpenalty\z@\exhyphenpenalty\z@
                \doublehyphendemerits\z@\finalhyphendemerits\z@
                \strut ##1\strut}%
        }%
        % If the linebreak is at a space, the latter will be displayed as visible
        % space at end of first line, and a continuation symbol starts next line.
        % Stretch/shrink are however usually zero for typewriter font.
        \def\FV@Space {%
            \nobreak\hskip\z@ plus\fontdimen3\font minus\fontdimen4\font
            \discretionary{\copy\Wrappedvisiblespacebox}{\Wrappedafterbreak}
            {\kern\fontdimen2\font}%
        }%
        
        % Allow breaks at special characters using \PYG... macros.
        \Wrappedbreaksatspecials
        % Breaks at punctuation characters . , ; ? ! and / need catcode=\active 	
        \OriginalVerbatim[#1,codes*=\Wrappedbreaksatpunct]%
    }
    \makeatother

    % Exact colors from NB
    \definecolor{incolor}{HTML}{303F9F}
    \definecolor{outcolor}{HTML}{D84315}
    \definecolor{cellborder}{HTML}{CFCFCF}
    \definecolor{cellbackground}{HTML}{F7F7F7}
    
    % prompt
    \makeatletter
    \newcommand{\boxspacing}{\kern\kvtcb@left@rule\kern\kvtcb@boxsep}
    \makeatother
    \newcommand{\prompt}[4]{
        \ttfamily\llap{{\color{#2}[#3]:\hspace{3pt}#4}}\vspace{-\baselineskip}
    }
    

    
    % Prevent overflowing lines due to hard-to-break entities
    \sloppy 
    % Setup hyperref package
    \hypersetup{
      breaklinks=true,  % so long urls are correctly broken across lines
      colorlinks=true,
      urlcolor=urlcolor,
      linkcolor=linkcolor,
      citecolor=citecolor,
      }
    % Slightly bigger margins than the latex defaults
    
    \geometry{verbose,tmargin=1in,bmargin=1in,lmargin=1in,rmargin=1in}
    
    

\begin{document}
    
    \title{Epidemiology of Economic Expectations}
    
    \author{Christopher Carroll \footnote{Department of Economics, Johns Hopkins University, \url{http://econ.jhu.edu/people/ccarroll/}, \url{ccarroll@jhu.edu.}} \\ Tao Wang \footnote{Department of Economics, Johns Hopkins University, \url{http://taowangecon.github.io}, \url{twang80@jhu.edu}.} }

\date{\today}
\maketitle
    
    ``A very natural next step for economics is to maintain expectations in
the strategic position they have come to occupy, but to build an
empirically validated theory of how attention is in fact directed within
a social system, and how expectations are, in fact, formed.''
\cite{simon_behavioral_1984}.

``While mass media play a major role in alerting individuals to the
possibility of an innovation, it seems to be personal contact that is
most relevant in leading to its adoption. Thus, the diffusion of an
innovation becomes a process formally akin to the spread of an
infectious disease.'' \cite{arrow_classificatory_1969}. 

``If we want to know why an unusually large economic event happened, we
need to list the seemingly unrelated narratives that all happened to be
going viral at around the same time and affecting the economy in the
same direction.'' \cite{shiller_narrative_2017}.

\hypertarget{outline}{%
\section{Outline}\label{outline}}

\hypertarget{motivation-and-context}{%
\subsection{Motivation and Context}\label{motivation-and-context}}

\begin{itemize}
\tightlist
\item
  Browning, Hansen and Heckman intro to ``Handbook of Macroeconomics'':
  \cite{browning_chapter_1999}. 

  \begin{itemize}
  \tightlist
  \item
    The most universal lesson of microeconomics is that ``people are
    different in ways that importantly affect their economic behavior''

    \begin{itemize}
    \tightlist
    \item
      circs: wealth, income
    \item
      prefs: risk aversion, impatience
    \end{itemize}
  \item
    Microfoundations of macro literature

    \begin{itemize}
    \tightlist
    \item
      When micro heterogeneity in circs or prefs is matched, fundamental
      conclusions change

      \begin{itemize}
      \tightlist
      \item
        Like, how do fiscal and monetary policy work
      \end{itemize}
    \end{itemize}
  \item
    Remaining kind of heterogeneity much less explored (until recently):
    in expectations/beliefs

    \begin{itemize}
    \tightlist
    \item
      even though heterogeneity in beliefs is just as apparent in micro
      data as other kinds
    \end{itemize}
  \item
    if expectations are heterogeneous, aggregate patterns depend on the
    distribution of expectations

    \begin{itemize}
    \tightlist
    \item
      stock market expectations of people who will never own stock are
      not important
    \item
      housing market expectations of the \emph{marginal} participants
      (buyer, seller) set prices \cite{piazzesi_momentum_2009} 
    \end{itemize}
  \item
    heterogenous expectations likely interact with other types of
    heterogeneity

    \begin{itemize}
    \tightlist
    \item
      rich people do more consumption than poor people
    \item
      so, need to weight expectations by the degree to which the
      person's actions affect the outcome
    \end{itemize}
  \item
    Existing literature on heterogeneity in expectations:

    \begin{itemize}
    \tightlist
    \item
      Most commonly explored reasons for heterogeneity in expectations:

      \begin{itemize}
      \tightlist
      \item
        information
      \item
        different updating (=``learning'') process
      \item
        different initial beliefs (=``priors'')
      \item
        different histories
      \item
        costs of updating info \(\Rightarrow\) frictions, delays
        \item
          even optimizing agents will happen to update at different
          times
      \end{itemize}
    \item
      Epidemiology is different from all of these

      \begin{itemize}
      \item
        It's about how ideas \emph{spread}
      \end{itemize}
    \end{itemize}
  \end{itemize}
\end{itemize}

\hypertarget{what-insights-can-the-epidemiological-framework-offer}{%
\subsection{What insights can the epidemiological framework
offer?}\label{what-insights-can-the-epidemiological-framework-offer}}

\begin{itemize}
\tightlist
\item
  Explains how people can have somewhat coherent beliefs about
  macroeconomy without PhD in macroeconomics

  \begin{itemize}
  \tightlist
  \item
    ordinary people: read news media to hear what ``experts'' say
  \item
    population beliefs depend on ``infectiousness'' of experts' views
  \item
    Embeds RE model as the limit corresponding to ``infinitely instantly
    perfectly infectious'' beliefs
  \item
    Epi models slow down the spread so it is testable

    \begin{itemize}
    \tightlist
    \item
      ``infectiousness'' matters

      \begin{itemize}
      \tightlist
      \item
        how social network and media affect the spread of economic news
        \cite{bailey_social_2018} \cite{cookson_why_2020}
      \item
        how policy communication makes its way to average economic
        agents making their decisions
      \end{itemize}
    \end{itemize}
  \item
    What should Fed governors, Treasury Secretaries, etc do 
\end{itemize}

\item and why it could be a useful tool, methodologically speaking 

\begin{itemize}
	\tightlist
	\item  if epi models are true, they have testable implications for cross-sectional
    distribution of expectations
  \item and implications for aggregate belief dynamics (and thus actual
    aggregate macro dynamics) 
   \item goal: reconcile micro/cross-section data   with (appropriately weighted) aggregate dynamics 
    \item goal: determine importance of social learning in structural models \cite{burnside_understanding_2016}
   
  \end{itemize}

\end{itemize}

\hypertarget{motivating-examples}{%
\subsection{Motivating examples}\label{motivating-examples}}

\begin{itemize}
\tightlist
\item
  household expectations for macroeconomic environment, i.e.~inflation
  \cite{carroll2001epidemiology}, \cite{nunes_epidemiological_2009}, 
  \cite{pfajfar_news_2013}
\item
  ponzi scheme and fraud \cite{akerlof_phishing_2016},
  \cite{mackay_extraordinary_2019}, \cite{rantala2019investment}
\item
  as a driver of finacial asset bubbles, especially some new class of
  assets, e.g.~bitcoin. \cite{shiller_narrative_2017}, \cite{kindleberger_manias_2011}
\item
  bank runs/spread of panic and fear

  \begin{itemize}
  \tightlist
  \item
    Canonical models are basically timeless: run happens instantly
    \cite{diamond_bank_1983}
  \item
    Understanding process by which they happen over time means
    possibility of arresting them
  \item
    Financial crisis in the Great Recession has been described as
    ``giant extended bank run on financial sector''
  \end{itemize}
\item
  housing prices \cite{burnside_understanding_2016}, \cite{piazzesi_momentum_2009}
\item
  stock investment \cite{barber_all_2008}
\end{itemize}

\hypertarget{epidemiology-model-basics}{%
\subsection{\texorpdfstring{Epidemiology model basics
\cite{anderson_infectious_1992}, \cite{kermack_contribution_1927}, 
\cite{hethcote_mathematics_2000}}{Epidemiology model basics  }}\label{epidemiology-model-basics}} 

\begin{itemize}
\item \emph{ex ante} homogeneous models 
\tightlist
\begin{itemize}
\item Common source 
\item SIS and SIR 
\item SEIR 
\item Features of homogenous models 
\begin{itemize}
	\item simplicity due to independence assumptions 
	\item easy to aggregate up to macro patterns and can be tested 
	\item capture the dynamics 
\end{itemize}
\end{itemize}

\item \emph{ex ante} heterogeneous
models (and consequences, if any) 
\begin{itemize}
	\item incorporates network structure \cite{jackson_social_2010} 
    \item  ``superspreaders'' 
    \begin{itemize}
    	\item higher degree distributions lead to more ``infection'' 
    	\item transition probability is location-specific depending on the degree of the node \item  interact with individual economic conditions 
    \end{itemize}
\end{itemize}
 \item   Economists' methodologies offer a lot of ways to improve standard epi models
 
 \begin{itemize}
 	\item transmission/infection rate can be contingent on state variables 
 	\item optimizing behavior by agents (incentives to seek/avoid infection) 
 	\item much more sophisticated about causality, inference, etc 
 	\item  possible that economists' ideas will infect the epidemiologists!
\end{itemize}
\end{itemize}

\hypertarget{potential-areas-where-techniques-could-be-applied}{%
\subsection{Potential areas where techniques could be
applied}\label{potential-areas-where-techniques-could-be-applied}}

\begin{itemize}
\tightlist
\item
economic sentiment and confidence \cite{carroll1994does}, \cite{benhabib2019sentiments}, \cite{mian_partisan_2018}
\item
  economic narratives \cite{shiller_narrative_2019}, \cite{lo_adaptive_2019}
\item
  spread of fake news and rumours 
  \cite{vosoughi_spread_2018},   \cite{dietz1967epidemics}
\item
  spread of default. \cite{schweikert2019epidemiological}
\item
  search-and-matching problems \cite{piazzesi_momentum_2009}
\item
  diffusion of innovation \cite{arrow_classificatory_1969}, 
  \cite{rogers_diffusion_2003}
\item
  communication, the role of opinion leaders \cite{iyengar_opinion_2010}
\item
  fads and fashions: information cascade \cite{bikhchandani_theory_1992}
\item
  viral marketing and internet memes, \cite{leskovec_dynamics_2007}, \cite{bauckhage_insights_2010} 
\end{itemize}

\hypertarget{relation-to-agent-based-modeling-in-macroeconomics}{%
\subsection{Relation to ``agent-based'' modeling in
macroeconomics}\label{relation-to-agent-based-modeling-in-macroeconomics}}

\begin{itemize}
\item Provides discipline on agents' actions/decisions   
\begin{itemize}
 \item \cite{lebaron_modeling_2008}
 \item  \cite{ragot_chapter_2018}
 \item  \cite{tesfatsion_chapter_2006}
 \item  \cite{haldane_drawing_2019}
 \item  In finance, \cite{lebaron_agent-based_2000}, \cite{lebaron_time_1999}
 \item  Both time-series and cross-sectional distribution properties
 \item  Allow examination of "off-equilirium" behavior \cite{simon_theories_1959}  
\end{itemize}
\end{itemize}

\hypertarget{other-toolboxes-to-use-and-promising-directions-of-research}{%
\subsection{Other toolboxes to use and promising directions of
research}\label{other-toolboxes-to-use-and-promising-directions-of-research}}

\begin{itemize}
\tightlist
\item
  Natural language processing \cite{gentzkow_text_2019}, \cite{cookson_why_2020}

  \begin{itemize}
  \tightlist
  \item
    counts of the word frequency

    \begin{itemize}
    \tightlist
    \item
      use google searchs to predict flu trends:\cite{dukic_tracking_2012}
    \end{itemize}
  \item
    ``sentiment analysis'' \cite{soo_quantifying_2015}
  \item
    topical modelling
  \item
    literature on differential infectiousness of different emotions

    \begin{itemize}
    \tightlist
    \item
      fear, anger, disgust more ``infectious'' than happiness,
      satisfaction
    \item
      e.g., could yield asymmetries between good and bad news
    \end{itemize}
  \end{itemize}
\item
  Social network data sources \cite{jackson_social_2010}

  \begin{itemize}
  \tightlist
  \item
    how network friends affect economic expectations
    \cite{bailey_social_2018}, \cite{bailey2018economic}

    \begin{itemize}
    \tightlist
    \item
      Potential application:

      \begin{itemize}
      \tightlist
      \item
        Greater geographical conectedness \(\Rightarrow\) fewer local
        boom/bust cycles, more aggregate ones
      \end{itemize}
    \end{itemize}
  \item
    social connectedness and aggregate economic outcomes
  \end{itemize}
\item
  Cognitive and neuroscience approach

  \begin{itemize}
  \tightlist
  \item
    ``microfoundation'' for results about differential infectiousness

    \begin{itemize}
    \tightlist
    \item
      example: FMRIs show brain's ``fear/disgust'' center more easily
      activated than pleasure/reward center
    \end{itemize}
  \end{itemize}
\end{itemize}

    % Add a bibliography block to the postdoc
   
\bibliographystyle{apalike}
\bibliography{EpiExpOutline}

    
\end{document}
