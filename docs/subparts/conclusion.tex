
\label{conclusion}%\hyperref{conclusion}{}

Many of the obstacles, real and perceived, to the construction of what we call full-fledged Epidemiological Expectations models have lessened over the last two decades.

A large body of evidence now finds that opinions on economic questions are sharply heterogeneous, and that people's choices are related to their opinions.

Data from social networks now provide the possibility of directly observing the operation of the key mechanisms of the social transmission of ideas -- as  has already been done in a few cases of economic models (and many more cases outside of economics).  Other work, based not on social network data but on measures like geographical proximity or shared workplaces or common places of origin, has found further evidence of social transmission of ideas, while another strand of research has explored the ways in which news outlets can be modeled as a source of heterogeneity in beliefs if news stories have degrees of either exposure or infectiousness less than 100 percent.

The recent successes achieved by the HA-Macro literature from the incorporation of realistic heterogeneity in non-expectational variables seem likely to tempt scholars to see what more can be accomplished by modeling expectational heterogeneity in ways based on empirically measured expectations.%  While there are other mechanisms for generating such heterogeneity, epidemiological modeling seems a promising candidate.

An epidemiological expectations modeling approach is by no means applicable only to macroeconomic questions -- the formation and consequences of expectations are at the heart of economic questions across the discipline.  A particularly attractive direction that any literature written by economists is likely to take is to apply the discipline's sophisticated tools for analyzing purposive behavior, as is done for example in the paper by~\cite{lucas2014knowledge} whose agents optimally expose themselves to the possibility of infection with new ideas in the hopes of improving their productivity -- something that scholars have done since time immemorial.

% (and outside it -- we have deliberately avoided dipping our toes into the vast literature on ``viral marketing'' -- for that see \cite{watts2007viral}).

\begin{comment}
\begin{itemize}
	\item Future directions of research
	\begin{itemize}
		\item Better measurements of market expectations/beliefs that could be fit into epi models.
		\begin{itemize}
			\item Explicitly eliciting people's source of information and reasons for held beliefs could shed light on the role of social connections on expectation formation.
		\end{itemize}
		\item Integrating observed social structure and interpersonal interactions with belief surveys to identify the epi models of expectations.
		\item Incorporating epi models of expectations into structural/general equilibrium models to examine if the belief dynamics has important implications for aggregate dynamics and outcomes
	\end{itemize}
\end{itemize}
\end{comment}
