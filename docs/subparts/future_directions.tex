\subsection{Future Directions}\label{subsec:future}\hypertarget{future}{}

Many suggestions for future work are contained in the foregoing.  Here, we mention a few further points that did not neatly fit above.

First, common tools from epidemiological practice could usefully be imported into expectational surveys -- particularly the deliberate efforts that epidemiologists make to determine the source of an infection (`contact tracing' being the most straightforward) -- usually by asking direct questions.  We would argue that, after a person's expectations have been elicited, at least a small amount of extra time should sometimes be allocated to asking ``why do you believe [x].''  The respondent may not be able to give an answer, but in many cases they might have a useful response: ``A friend told me'' or ``I read it in the newspaper'' or ``I did some research on the internet'' or ``I learned that from my parents when I was growing up.''\footnote{\cite{arrondel2020informative} provides an example of this approach. In a survey of French households, they not only elicited respondents' stock market expectations, but also the size and financial expertise of the social circles within which they discuss financial matters. The paper finds that social interactions affect stock market beliefs mostly through information channels, instead of social  preferences.} Any of these answers (or potentially others) might prove very helpful in narrowing the set of models that are plausible for explaining any particular set of beliefs.  Direct questions could also help distinguish between different kinds of information: A job seeker might learn from friends that job prospects have improved, which causes improved expectations. These same friends might also tell the job-seeker about a specific current vacancy.  If job seekers were directly asked separate questions about expectations and vacancy tips, it would be much easier to distinguish a mechanism in which optimistic job-seekers work harder to find jobs from a mechanism in which job vacancy tips are more frequent in periods when optimism is greater.

Above, we made mention of several kinds of evidence that information from some sources, or of some kinds, was more infectious.  The literature on homophily, for example, suggests that ideas spread more readily among persons who have more in common.  And even among social connections with otherwise-similar characteristics, some are likely to be more credible than others.  %({\it ceteris paribus}, stock advice from an investment banker cousin might be more likely to be trusted than from his identical twin who is a circus acrobat.)
Direct survey questions asking respondents which sources of information they find most persuasive, and why, might prove very helpful in thinking about the most appropriate structure for our models (and potentially even addressing problems like the Manski's reflection problem.)

Epidemiological ideas might also prove to be useful in understanding how to interpret results like those in \cite{galesic2018asking}, who find that election surveys that ask participants about the voting intentions of their social contacts proved more accurate in predicting voting outcomes than surveys asking people how they themselves would vote, and other results that suggest that it is easier to elicit prevalence of socially stigmatized behaviors or attitudes by asking respondents about prevalence among members of their social circles rather than asking the respondent about themselves.  (``Are you a racist'' does not elicit useful responses - and may even result in the termination of the survey; ``how many of your friends would you say might be racist'' seems to generate much more revealing responses; cf \cite{radas2021predicted}.)