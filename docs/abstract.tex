  `Epidemiological' models of belief formation put social interactions at their core; such models are widely used by scholars who are not economists to study the dynamics of beliefs in populations.  We survey the literature in which economists attempting to model the consequences of beliefs about the future -- `expectations' -- have employed a full-fledged epidemiological approach to explore an economic question.  We draw connections to related work on `contagion,' narrative economics, news/rumor spreading, and the spread of internet memes. A main theme of the paper is that a number of independent developments have recently converged to make epidemiological expectations (`EE') modeling more feasible and appealing than in the past.
